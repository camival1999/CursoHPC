\documentclass[11pt,letterpaper]{exam}
\usepackage[utf8]{inputenc}
%\usepackage[spanish]{babel}
\usepackage{graphicx}
\usepackage{float}
%\decimalpoint

\begin{document}
\begin{center}
{\Large Introducción a la Computaciuón Científica de Alto Rendimiento} \\
S4C1 tarea - \textsc{Error numérico}\\
03-2023\\
Camilo Andres Valencia Acevedo\\
cvalenciaa@unal.edu.co
\end{center}


\section{Explicación}
En este trabajo se buscó obtener una aproximación a la derivada de la función $sin(x)$ por medio de dos métodos numéricos distintos empleando la definición de límite en ambos casos: en la figura \ref{fig:forward} por medio de pasos $h$ completos hacia adelante o \textit{Forward}, y en la figura \ref{fig:central} tomando medio paso hacia atrás y otro medio adelante o \textit{Central}.
Tras ello se mostrará la convergencia de la aproximación numérica tipo \textit{Central} por medio de la figura \ref{fig:convergencia}.

\noindent
\section{Gr\'aficas}
Aquí podemos apreciar los resultados de aplicar ambas técnicas:
\begin{figure}[H]
    \centering
    \includegraphics[width=10cm]{err_derF.pdf} 
    \caption{Error absoluto por el método Forward.}
    \label{fig:forward}
\end{figure}

\begin{figure}[H]
    \centering
    \includegraphics[width=10cm]{err_derC.pdf} 
    \caption{Error absoluto por el método Central.}
    \label{fig:central}
\end{figure}

\begin{figure}[H]
    \centering
    \includegraphics[width=10cm]{err_der_h.pdf} 
    \caption{Convergencia del valor de $Cos(x)$ según el tamaño de paso.}
    \label{fig:convergencia}
\end{figure}

Es posible observar que de ambas aproximaciones numéricas, el método \textit{Central} es considerablemente mejor obteniendo un error constante muy bajo, mientras que el método de \textit{Forward} posee un alto porcentaje de error para valores alejados del 0, a pesar de tener un buen desempeño cercano a este.
Adicionalmente respecto a la convergencia mostrada en la figura \ref{fig:convergencia}, podemos reafirmar nuevamente la eficacia del método \textit{Central}, pues incluso con un valor de paso tan elevado como es \textit{1}, es capaz de obtener una muy buena aproximación al valor real deseado, y rápidamente converge con error despreciable a este con un $h$ cercano a $0.1$

\end{document}
