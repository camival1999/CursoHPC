\documentclass[11pt,letterpaper]{exam}
\usepackage[utf8]{inputenc}
%\usepackage[spanish]{babel}
\usepackage{graphicx}
\usepackage{float}
%\decimalpoint

\begin{document}
\begin{center}
{\Large Introducción a la Computación Científica de Alto Rendimiento} \\
S5C1 tarea - \textsc{Manejo de librerías estándar}\\
03-2023\\
Camilo Andres Valencia Acevedo\\
cvalenciaa@unal.edu.co
\end{center}


\section{Explicación}
En este trabajo se buscó obtener un acercamiento a las librerías estándar de C++ mediante la creación de una serie de arreglos que contienen datos enteros y flotantes los cuales siguen determinadas distribuciones estadísticas.
Podemos observar los resultados de ambas en las figuras \ref{fig:int} y \\ref{fig:float} para la distribución uniforme de enteros entre 0 y 10, y Gaussiana centrada en -10 con desviación estándar de 17 de flotantes respectivamente.

\noindent
\section{Gr\'aficas}
    \begin{figure}[H]
        \centering
        \includegraphics[width=10cm]{histogramaInt.pdf} 
        \caption{Histograma de datos enteros.}
        \label{fig:int}
    \end{figure}

    \begin{figure}[H]
        \centering
        \includegraphics[width=10cm]{histogramaFloat.pdf} 
        \caption{Histograma de datos flotantes.}
        \label{fig:float}
    \end{figure}

\end{document}
